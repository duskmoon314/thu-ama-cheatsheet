\documentclass[a4paper,10pt]{ctexart}
\usepackage[left=.5cm, right=.5cm, top=.5cm, bottom=.5cm]{geometry}
\usepackage{multicol}

\usepackage{amsmath,amssymb,amsthm}
\usepackage{mathrsfs}
\usepackage{thmtools}
\usepackage{stmaryrd}

\usepackage{booktabs}
\usepackage[shortlabels]{enumitem}

\usepackage[colorlinks]{hyperref}
\usepackage{fontawesome5}

\usepackage{tikz-cd}

\usepackage{xargs}

% ctex 排版设置
\ctexset{
    section = {
        beforeskip = {-1ex},
        afterskip = {-1ex},
    },
    subsection = {
        beforeskip = {-1ex},
        afterskip = {-1ex},
    },
    subsubsection = {
        beforeskip = {-1ex},
        afterskip = {-1ex},
    },
}
\setlength{\parindent}{0pt} % 抑制首行缩进以减少空白
\setlist{nosep} % 抑制列表项之间的空白

% 数学符号
\newcommand*{\bb}[1]{\mathbb{#1}}
\newcommand*{\setN}{\mathbb{N}}
\newcommand*{\setZ}{\mathbb{Z}}
\newcommand*{\setQ}{\mathbb{Q}}
\newcommand*{\setR}{\mathbb{R}}
\newcommand*{\setC}{\mathbb{C}}
\newcommand*{\st}{\text{ s.t. }}
\newcommand*{\stab}{\text{stab}}
\newcommand*{\powerset}[1]{\mathscr{P}(#1)}
\newcommand*{\impl}{\Rightarrow}
\renewcommand*{\iff}{\Leftrightarrow}
\renewcommand*{\leq}{\leqslant}
\renewcommand*{\geq}{\geqslant}
\newcommand*{\nmsubgroupeq}{\trianglelefteqslant}
\newcommand*{\nmsubgroup}{\triangleleft}
\newcommand*{\gengroup}[1]{\langle #1 \rangle}
\newcommand*{\genring}[1]{[#1]}
\newcommand*{\genideal}[1]{(#1)}
\newcommand*{\genfield}[1]{\langle #1 \rangle}
\newcommand*{\ch}[1]{\text{ch}(#1)}
\newcommand*{\set}[1]{\{#1\}}
\newcommandx*\homo[3][3=]{#1 \overset{#3}{\rightarrow} #2}
\newcommandx*\mono[3][3=]{#1 \overset{#3}{\hookrightarrow} #2}
\newcommandx*\epi[3][3=]{#1 \overset{#3}{\twoheadrightarrow} #2}
\newcommandx*\iso[3][3=]{#1 \overunderset{\sim}{#3}{\rightarrow} #2}
\newcommand*{\ff}[1]{\bb{F}_{#1}}

% 数学环境
\declaretheoremstyle[
    spaceabove=0pt, spacebelow=0pt,
    headindent=-4pt,
    notefont=\bfseries,notebraces={}{}
]{tight-thm}
\declaretheorem[name=, numbered=no, style=tight-thm]{theorem}

\begin{document}

\begin{multicols}{3}
    \begin{center}
        {\Large 应用近世代数}

        暮月

        \vspace*{-0.5em}

        \href{https://github.com/duskmoon314/thu-ama-cheatsheet}{\faGithub \  duskmoon314/thu-ama-cheatsheet}

        \vspace*{-1.5em}
    \end{center}

    \section{预备知识}

    \subsection{集合与映射}

    \textbf{常见数集} $\setN, \setZ, \setQ, \setR, \setC$

    \textbf{幂集} A所有子集构成集合,$\powerset{A}$或$2^A$

    \textbf{映射} $f: A \to B, x \mapsto f(x) \in B$

    \begin{itemize}
        \item[\emph{单射}] $\forall x_1, x_2 \!\in\! A, x_1 \!\neq\! x_2 \!\impl\! f(x_1) \!\neq\! f(x_2) $
        \item[\emph{满射}] $f(A) \! = \! B \impl \forall y \!\in\! B, \exists x \!\in\! A, f(x) = y$
        \item[\emph{双射}] 单射且满射 $\impl$ 可逆
    \end{itemize}

    \textbf{变换} $f: A \to A$,$f$是A上的变换

    \textbf{复合映射} $fg(x) = f(g(x))$

    \textbf{恒等变换} $I_A(x) = x$

    \textbf{映射的逆} $gf \! = \! I_A, fh \! = \! I_B \impl g \! = \! h \!= \! f^{-1}$

    \subsection{二元关系}

    \textbf{二元运算} $f: S \times S \to S$

    \textbf{二元关系} $\forall a \!\in\! A, b \!\in\! B$可判$\!aRb\!$或$\!aR'b\!$

    \textbf{等价关系} $\sim$是等价关系$\iff$满足

    \begin{itemize}
        \item[\emph{反身}] $\forall a \!\in\! A, a \sim a$
        \item[\emph{对称}] $\forall a, b \!\in\! A, a \sim b \impl b \sim a$
        \item[\emph{传递}] $\forall a, b, c \!\in\! A, a \sim b, b \sim c \impl a \sim c $
    \end{itemize}

    \textbf{等价类} $\bar{a} = \{ x | x \in A, x \sim a \}$ \textbf{代表元}$a$

    \textbf{划分} $\{ A_\alpha | \alpha \in I \}$ 是划分$\iff$满足

    \begin{itemize}
        \item $\bigcup\limits_{\alpha \in I} A_\alpha = A$
        \item $\forall \alpha, \beta \in I, \alpha \ne \beta, A_\alpha \bigcap A_\beta = \emptyset$
    \end{itemize}

    \begin{theorem}[等价与划分]
        $a \sim b \iff \exists \alpha \in I, a, b \in A_\alpha$
    \end{theorem}

    \textbf{商集} $A/\sim = \{ \bar{a} | a \in A \}$

    \textbf{偏序} 反身、反对称、传递

    \textbf{全序} 偏序且$\forall x, y \in S, x \leq y$或$y \leq x$

    \subsection{整数与同余}

    \begin{theorem}[Bézout定理]
        $a, b \in \setZ^*, \exists p, q \in \setZ \st pa + qb = \gcd(a, b) = d$
    \end{theorem}

    \begin{theorem}[gcd \& lcm]
        $(a, b)[a, b] = ab$
    \end{theorem}

    \textbf{一次同余方程} $ax \!\equiv\! b \pmod{m} \iff (a, m) \!\mid\! b$\\
    通解$x \equiv pb_1 \pmod{m_1}$ $(a = a_1(a, m), b = b_1(a, m), m = m_1(a, m), pa_1 +qm_1 = 1)$

    \textbf{一次同余方程组} $\begin{cases}
            x \equiv b_1 \pmod{m_1} \\
            x \equiv b_k \pmod{m_k}
        \end{cases}$\\
    通解$x \equiv \sum_{i=1}^k b_i c_i M_i \pmod{M}$\\
    $M \! = \! \prod\limits_{i=1}^k m_i, M_i \! = \! \frac{M}{m_i}, M_i c_i \equiv 1 \pmod{m_i}$

    \section{群论}

    \subsection{概念与例子}

    \textbf{群} $(G, \cdot)$ 封结幺逆(交)

    \begin{theorem}[幺元性质]
        $e_L = e_R = e$
    \end{theorem}

    \begin{theorem}[逆元性质]
        $\forall a \in G, a_L^{-1} = a_R^{-1} = a^{-1}$

        \begin{itemize}
            \item $(a^{-1})^{-1} = a$
            \item $(ab)^{-1}=b^{-1}a^{-1}$
            \item $(a^n)^{-1}=(a^{-1})^n=a^{-n}$
        \end{itemize}
    \end{theorem}

    \subsection{内部结构}

    \subsubsection{子群}

    \textbf{子群} $\emptyset \!\neq\! H \!\subseteq\! G$,封结幺逆$\impl H \!\leq\! G$

    \textbf{平凡子群} $\{e\}, G$

    \begin{theorem}[子群判则]
        以下等价
        \begin{enumerate}
            \item $H \leq G$
            \item $\forall a, b \!\in\! H$,$ab \!\in\! H, a^{-1} \!\in\! H$ (四合二)
            \item $\forall a, b \!\in\! H$,$ab^{-1} \!\in\! H$ (四合一)
        \end{enumerate}
    \end{theorem}

    \begin{theorem}[子群运算律($H_1, H_2 \!\leq\! G$)]
        \hfil

        \begin{enumerate}
            \item $H \!\leq\! G$,则$H$中幺元为$G$中幺元
            \item $H_1 \!\cap\! H_2 \!\leq\! G$
            \item $H_1 \!\cup\! H_2 \!\leq\! G \!\iff\! H_1 \!\subseteq\! H_2 $或$H_2 \!\subseteq\! H_1$
            \item $H_1 H_2 \!\leq\! G \!\iff\! H_1 H_2 = H_2 H_1 \\ \!\triangleq\! \{h_1h_2 | h_1 \!\in\! H_1, h_2 \!\in\! H_2\}$
        \end{enumerate}
    \end{theorem}

    \textbf{元素的阶} $a \!\in\! G$,最小$n$使$a^n \!=\! e$,记$o(a)$

    \begin{theorem}[阶的性质]
        \hfil

        \begin{enumerate}
            \item $o(e) = 1$, $o(a^{-1}) = o(a)$
            \item $a^m = e \iff o(a) | m$
            \item 有限群中所有元素的阶均有限,而无限群不一定存在无限阶元
            \item 设$o(a) = m, o(b) = n$,若$(m, n) = 1$且$ab = ba$,则$o(ab) = mn$
            \item $\forall a \!\in\! G \!\setminus\! \{e\}$,若$o(a) = 2$,G交换群
        \end{enumerate}
    \end{theorem}

    \subsubsection{生成子群 \& 循环群}

    \textbf{生成子群} $S \!\subseteq\! G$,$G$中含$S$的子群交,$\gengroup{S}$

    \textbf{循环子群} 由1个元素生成的子群

    \textbf{极小生成集} $G \! = \! \gengroup{S}$,且任何S的真子集的生成子群均不是G,元素个数唯一

    \begin{theorem}[循环群性质]
        \hfil

        \begin{enumerate}
            \item 循环群必交换
            \item 循环子群的阶等于生成元的阶
            \item $\gengroup{a} \cong \begin{cases}
                          (\setZ, +)        & o(a)=\infty       \\
                          (\setZ/n\setZ, +) & o(a) = n < \infty \\
                      \end{cases}$
            \item 阶数相同的循环群必同构
            \item $(\setZ, +)\!$子群必为$m\setZ(m \!\in\! \setZ)$形式

                  $(\setZ/n\setZ,\! +)\!$子群同构于$\!(\setZ/d\setZ,\! +)\  d|n$
        \end{enumerate}
    \end{theorem}

    \subsubsection{置换群}

    \textbf{变换群\ } 非空集合$\Omega$上所有可逆变换全体

    \textbf{置换群(对称群)} 有限集上的变换群$S_n$

    \textbf{置换的共轭} $\tau \sigma \tau^{-1} = \sigma_1$

    \textbf{轮换} $\sigma(i_k) = i_{((k+1))}$,其余不动点

    \textbf{对换(循环)} 长度为2的轮换

    \begin{theorem}[标准轮换分解]
        $\forall \sigma \in S_n$,可分解为唯一不相交轮换之积$r_1 r_2 \cdots r_s$
    \end{theorem}

    \textbf{奇(偶)置换} $\sigma$可表为奇(偶)个对换之积

    \textbf{交错群} $A_n = \{ S_n \text{中全体偶置换} \}$

    \begin{theorem}[置换群性质]
        任意置换群全偶或奇偶各半
    \end{theorem}

    \begin{theorem}[Cayley定理]
        任一群$G$必同构于左乘变换群。有限群则$G \!\cong\! G' \!\leq\! S_n$
    \end{theorem}

    \subsubsection{子群陪集}

    \textbf{左(右)陪集} $\forall a \!\in\! G, H \!\leq\! G, aH(Ha)$

    \begin{theorem}[陪集相等判则]
        $aH = bH \iff a^{-1}b \in H$
    \end{theorem}

    \textbf{指数} 左(右)陪集的个数,记$[G\!:\!H](H \!\leq\! G)$

    \begin{theorem}[Lagrange定理]
        $H \!\leq\! G \!\impl\! |G| \! = \! |H|[G:H]$
    \end{theorem}

    \begin{theorem}[指数性质]
        \hfil

        \begin{enumerate}
            \item $|G| < \infty, |H| \mid |G|, \forall a \in G, o(a) \mid |G|$
            \item $K \!\leq\! H \!\leq\! G \!\impl\! [G\!:\!K] \! = \! [G\!:\!H][H\!:\!K]$
        \end{enumerate}
    \end{theorem}

    \begin{theorem}[子群阶]
        $A, B \!\leq\! G$且有限$\impl |AB| = \frac{ |A| |B| }{ |A \cap B| }$
    \end{theorem}

    \subsubsection{正规子群 \& 商群}

    \textbf{正规子群} $H \!\!\leq\!\! G, \forall g \!\in\! G, gH \! = \! Hg$记$H \!\nmsubgroupeq\! G$

    \textbf{单群\ } 不存在非平凡正规子群的群

    \textbf{典例\ } 指数为2的子群必正规

    \begin{theorem}[正规子群判则]
        $H \leq G$,以下等价
        \begin{enumerate}
            \item $H \nmsubgroupeq G$
            \item $\forall g \in G,\forall h \in H, ghg^{-1} \in H$
            \item $\forall g \in G, gHg^{-1} \subseteq H$
            \item $\forall g \in G, gHg^{-1} = H$
        \end{enumerate}
    \end{theorem}

    \begin{theorem}[正规子群运算律]
        \hfil

        \begin{enumerate}
            \item $A \!\nmsubgroupeq\! G, B \!\nmsubgroupeq\! G \!\impl\! A \!\cap\! B \!\nmsubgroupeq\! G, AB \!\nmsubgroupeq\! G$
            \item $A \!\nmsubgroupeq\! G, B \!\leq\! G \!\impl\! A \!\cap\! B \!\nmsubgroupeq\! B, AB \!\leq\! G$
            \item $A, B \!\nmsubgroupeq\! G, A \!\cap\! B \! = \! \{e\} \!\impl\! ab \! = \! ba \\ (\forall a \!\in\! A, b \!\in\! B)$
        \end{enumerate}
    \end{theorem}

    \textbf{换位子} $aba^{-1}b^{-1}$,记作$[a,b]$

    \textbf{换位子群} $\{ \gengroup{aba^{-1}b^{-1}} | a, b \in G \} \nmsubgroupeq G$

    \textbf{商群} $H \!\nmsubgroupeq\! G$,$G/H = \{ aH | a \in G \}$

    \begin{theorem}[素阶元]
        有限交换$G, p||G|$,$G$必有$o(x) \! = \! p$
    \end{theorem}

    \subsubsection{中心化子 \& 共轭子群}

    \textbf{中心} $C(G) = \{ x \in G | xg = gx, \forall g \in G\}$

    \textbf{中心化子} $C_G(A) \! = \! \{x \!\in\! G | xa \! = \! ax, \forall a \!\in\! A\}$

    \begin{theorem}[中心化子性质]
        \hfil

        \begin{enumerate}
            \item $C_G(a) \cap C_G(b) = C_G(\{a, b\})$
            \item $\gengroup{a} \leq C_G(a)$
            \item $\{e\} \!\leq\! C(G) \!\leq\! C_G(a) \!\leq\! C_G(A) \!\leq\! G$
        \end{enumerate}
    \end{theorem}

    \textbf{共轭} $a, b \in G, \exists g \in G \st gag^{-1}=b$

    \textbf{共轭类} $K_a = \{ gag^{-1} | g \in G \}$

    \begin{theorem}[共轭子群基数定理]
        $|K_a| = [G:C_G(a)]$
    \end{theorem}

    \begin{theorem}[类方程]
        $|G| \! < \! \infty, |G| \! = \! |C| + \sum\limits_{a \notin C} [G:C_G(a)]$
    \end{theorem}

    \textbf{共轭子群} $H \leq G, K = gHg^{-1}$

    \textbf{正规化子} $N_G(H) = \{g \in G | gH = Hg\}$

    \textbf{共轭子群类} $K_H = \{gHg^{-1} | g \in G\}$

    \begin{theorem}[共轭子群类基数定理]
        $|K_H| \!\! = \!\! [G:N_G(H)]$
    \end{theorem}

    \begin{theorem}[类方程2]
        $|\mathcal{A}| = |\mathcal{N}| + \sum\limits_{H \notin \mathcal{N}} [G:N_G(H)]$ \\
        $\mathcal{A} = \{ H | H \leq G \}, \mathcal{N} = \{ H | H \nmsubgroupeq G \}$
    \end{theorem}

    \subsection{外部联系(群同态)} \label{sec:group-homomorphism}

    \textbf{同态} \ 映射 + 保运算

    \textbf{单同态} \ 单射 + 同态 $G \overset{f}{\hookrightarrow} G'$

    \textbf{满同态} \ 满射 + 同态 $G \overset{f}{\twoheadrightarrow} G'$

    \textbf{同构} \ 双射 + 同态 $G \overunderset{\sim}{f}{\rightarrow} G'$

    \textbf{核} $\ker f = \{ a \in G | f(a) = e' \} \nmsubgroupeq G$

    \begin{theorem}[同态性质]
        \hfil

        \begin{enumerate}
            \item 保幺元、逆元、方幂
            \item 保子群 $H \leq G \impl f(H) \leq G'$\\
                  $H \!\nmsubgroupeq\! G \!\impl\! f(H) \!\nmsubgroupeq\! f(G)$
            \item 反向局部保(正规)子群 \\
                  $N \!\leq\! f(G) \!\impl\! f^{-1}(N) \!\leq\! G$
            \item 保阶为因子 $o(a) \!<\! \infty \!\impl\! o(f(a)) | o(a)$
            \item $f(a) = a' \impl f^{-1}(a') = a \cdot \ker f$
            \item $f$单同态$\impl \ker f = \{e\}$
        \end{enumerate}
    \end{theorem}

    \begin{theorem}[同态基本定理]
        $f: G \rightarrow G', K = \ker f$

        \begin{enumerate}
            \item $G/K \cong f(G)$,若$f$满,则$G/K \cong G'$
            \item $\pi: G \twoheadrightarrow G/K, j: f(G) \hookrightarrow G' \impl f = j \circ \bar{f} \circ \pi, \bar{f}: G/K \rightarrow f(G), \bar{a} \mapsto f(a)$
        \end{enumerate}

        \begin{tikzcd}[column sep = tiny, row sep = small]
            G \arrow[rr, "f", two heads] \arrow[dr, "\pi"', two heads] && {G'}  \\
            & {G/K} \arrow[ur, "\sim", "\bar{f}"']
        \end{tikzcd}
        \begin{tikzcd}[row sep = small]
            G \arrow[r, "f"] \arrow[d, "\pi"', two heads] & {G'} \\
            {G/K} \arrow[r, "\sim", "\bar{f}"'] & {f(G)} \arrow[u, "j"', hook]
        \end{tikzcd}
    \end{theorem}

    \begin{theorem}[子群对应定理]
        $f: \epi{G}{G'}, K = \ker f, S = \{ H | K \leq H \leq G \}, S' = \{ N | N \leq G' \}$,则$S$与$S'$一一对应
    \end{theorem}

    \begin{theorem}[商群同构定理(第一同构定理)]
        \hfil

        $f: \epi{G}{G'}, K = \ker f \leq H \nmsubgroupeq G \impl f(H) \nmsubgroupeq G', G/H \cong G'/f(H) \cong \frac{G/K}{H/K}$
    \end{theorem}

    \begin{theorem}[子群乘积同构定理(第二同构定理)]
        \hfil

        $N \nmsubgroupeq G, H \leq G \impl HN/N \cong H/(H \cap N)$
    \end{theorem}

    \begin{theorem}[推论]
        \hfil
        \begin{enumerate}
            \item $N \!\nmsubgroupeq\! H \!\nmsubgroupeq\! G, N \!\nmsubgroupeq\! G \!\impl\! \frac{G/N}{H/N} \!\cong\! G/H$
            \item $N$是极大正规子群$\iff G/N$是单群
        \end{enumerate}
    \end{theorem}

    \subsection{群在集合上的作用}

    \textbf{群作用} $\forall g \in G \iff \Omega$上$g(x)$且单位、相容

    \textbf{轨道} $\Omega_x = \{ g(x) | g \in G \}$,代表元$x$

    \textbf{不动点} $g(x) = x$

    \textbf{可迁} $\forall x, y \in \Omega, \exists g \in G, g(x) = y$

    \textbf{稳定化子} $G_x \! = \! \{ g \!\in\! G | g(x) \! = \! x \} \!\leq\! G$

    \textbf{轨道公式} $|\Omega_x| = |G : G_x|, |G| = |\Omega_x||G_x|$

    \begin{theorem}[Burnside 引理]
        $G$作用$\Omega$,轨道数$N$,$g$在$\Omega$中不动点数$\chi(g)$,则$N = \frac{1}{|G|} \sum\limits_{g \in G} \chi(g)$
    \end{theorem}

    \section{环论}

    \subsection{概念与例子}

    \textbf{环} $(R, +, \cdot)$ 加交换群 + 乘半群 + 分配律

    \textbf{零元} $0 \cdot a = a \cdot 0 (\forall a \in R)$

    \textbf{负元} $(-a)b = a(-b) = -(ab) (\forall a, b \in R)$

    \textbf{单位} $U(R)$ 可逆元(必非零因子)

    \textbf{零因子} $a, b \in R, ab = 0 \wedge a, b \ne 0$

    \textbf{整环} $R\ne \{0\}$,含幺可交换无零因子

    \textbf{除环} $\{0, 1\} \subseteq R, R \setminus \{0\}$构成乘法群

    非零有限无零因子是除环$\!\impl\!$有限整环是域

    \subsection{内部结构}

    \subsubsection{子环、理想和商环}

    \textbf{子环} $\emptyset \!\neq\! S \!\subseteq\! R \wedge (S, +, \cdot)$为环,$S \leq R$

    \begin{theorem}[子环判则]
        $\forall a, b \in S, a - b, ab \in S$
    \end{theorem}

    \begin{theorem}[子环运算律]
        $S_1, S_2 \leq R$
        \begin{enumerate}
            \item $S_1 \cap S_2 \leq R$
            \item $S_1 \cup S_2, S_1 + S_2, S_1S_2$不一定
        \end{enumerate}
    \end{theorem}

    \textbf{左(右)理想} $\forall a \in I, r \in R, ra \in I (ar \in I)$

    $R$含幺,$I \! = \! R \iff 1 \!\in\! I \iff u \!\in\! I (\forall u \!\in\! U(R))$

    \begin{theorem}[理想判则]
        $\forall a, b \!\in\! I, \forall r \!\in\! R, ra, ar, a-b \in I$
    \end{theorem}

    \begin{theorem}[理想运算律]
        $I, J$是,$I \cap J, I + J, IJ$也是
    \end{theorem}

    \textbf{单环\ } 只有平凡理想 $\{0\}$ 和 $R$ 的环

    \textbf{生成子环} $\emptyset \!\neq\! S \!\subseteq\! R, R$中含$\!S\!$的子环交$\genring{S}$

    \textbf{生成理想} $\emptyset \!\neq\! S \!\subseteq\! R, R$中含$\!S\!$的理想交$\genideal{S}$

    \textbf{主理想\ } 由1个元素生成的理想

    \textbf{素理想} $P \nmsubgroupeq R$换,$ab \in P \impl a \in P \vee b \in P$

    \textbf{商环} $R$和$I$的加法商群引入乘法,$R/I$

    \textbf{极大理想} $M$非平凡,无更大非平凡理想

    $R$含幺交换,$M$极大$\iff R/M$是域

    \subsection{外部联系(环同态)}

    参见群同态(\ref{sec:group-homomorphism})

    \subsection{特殊环}

    \textbf{因子、倍元、整除} $\exists c \in D, a = bc, b | a$

    \textbf{相伴} $a \sim b: a | b, b | a$

    \textbf{真因子} $a = bc, b, c \notin U(D)$

    \textbf{既约元} $p \in D \setminus (\{ 0 \} \cup U(D))$,无真因子

    \textbf{素元} $p | ab \impl p | a \vee p | b$

    含幺整环$D$中素元必为既约元,反之不然

    \textbf{gcd} $d \sim (a, b): d|a \wedge d|b \wedge \forall c, c|a \wedge c|b \impl c|d$

    \begin{theorem}[gcd性质]
        \hfil

        \begin{enumerate}
            \item $(a, (b, c)) \sim ((a, b), c)$
            \item $c(a, b) \sim (ca, cb)$
            \item \textbf{互素} $(a, b) \sim 1$
            \item GCD条件 $\impl$ 素性条件
        \end{enumerate}
    \end{theorem}

    \textbf{GCD条件} $D$中任两不全0元素均有gcd

    \textbf{素性条件} $D$中的每个既约元也是素元

    \textbf{因子链条件} $D$中任真因子链只有有限项

    \textbf{UFD} $\forall a \!\in\! D \setminus (\{ 0 \} \!\cup\! U(D)), a \! = \! \prod p_i^{\epsilon_i}$唯一\\
    UFD $\iff$ GCD + 素性 $\iff$ 素性 + 因子链

    \textbf{PID} $D$的任意理想都是主理想 $\subset$ UFD

    \textbf{ED 范数} $\exists v: \homo{D^*}{\setZ^+}, \forall a, b \!\in\! D, \exists q, r \!\in\! D \st b \! = \! aq + r, r \! = \! 0 \!\vee v(r) \!<\! v(a) \subset$ PID

    \textbf{本原多项式} $\varphi(x) \!\ne\! 0 \!\in\! D[x], (a_0 \cdots a_n) \sim 1$

    \textbf{有理根判则} $f(x) \in D[x], \frac{r}{s}$满足$f(\frac{r}{s}) = 0, (r, s) \sim 1$,则$r | a_0, s | a_n$

    \textbf{Eisenstein判则} $f(x) \!\in\! D[x], p$满足$p \!\nmid\! a_n, p | a_i (0 \!\leq\! i \!<\! n), p^2 \nmid a_0$,则$f(x)$不可约

    \section{域论}

    \subsection{域扩张}

    \textbf{子域/扩域} $\emptyset \ne F \subseteq K$,皆为域

    \textbf{生成子域} $S \subseteq F$,$F$中含$S$最小子域,$\genfield{S}$

    \textbf{素域} $F_0 = \genfield{1}$

    \textbf{特征} $\ch{F} = \begin{cases}
            p & \text{若}o^+(1) = p < \infty \\
            0 & \text{若}o^+(1) = \infty
        \end{cases}$

    \begin{theorem}[素域同构]
        $F_0$是$F$的素域,

        $F_0 \cong \begin{cases}
                \setQ        & o^+(1) = \infty     \\
                \setZ/p\setZ & o^+(1) = p < \infty
            \end{cases}$
    \end{theorem}

    \textbf{扩张次数} $K/F$,$K$作为$F$-线性空间的维数$\dim_F K$,$(K:F)$

    \begin{theorem}[链式法则]
        $F\subseteq K \subseteq E, E/F$有限扩张$\iff (E:F)=(E:K)(K:F)$
    \end{theorem}

    \textbf{添元扩张} $K/F, S\subseteq K$,含$S\cup F$的$K$最小子域为$F(S)$

    \textbf{单扩张} $S = \set{u}, F(u)$

    \textbf{代数元(超越元)} $K/F, u\in K, u$是(否) $F$上某非零多项式$f(x)\in F[x]$根

    \textbf{最小多项式} $F$上$u$为根首一次数最小($n$)多项式$m_u(x), u$为$n$次代数元

    \begin{theorem}[单扩张定理]
        $E/F, u \in E$,则\\
        $F(u) \cong \begin{cases}
                \{ \sum a_i u^i | a_i \in F \} \cong F[x]/(m_u(x)) \\
                \{ \frac{f(x)}{g(x)} | f,g \in F[x] \} \cong F(x)
            \end{cases}$\\
        $(F(u)\!:\!F) = \deg m_u(x) (\text{代数元}), \infty (\text{超越元})$
    \end{theorem}

    \textbf{代数扩张(超越扩张)} $K/F, u \in K$均为代数元,否则(只须1个)为超越扩张

    \begin{theorem}[代数扩张性质]
        \hfil

        \begin{enumerate}
            \item $a, b \in E/F, (F(a)\!:\!F) \! = \! m, (F(b)\!:\!F) \! = \! n, (F(a,b)\!:\!F) \!\leq\! mn$
            \item $a,b \!\in\! E/F$为$F$代数元,四则运算\checkmark
            \item 有限扩张$\impl$代数扩张,反之不然
            \item $K/F, E/K$代数扩,$E/F$代数扩
        \end{enumerate}
    \end{theorem}

    \textbf{代数闭包} $F$扩全体代数元,$F^{AC}$

    \subsection{有限域}

    \textbf{构造} $p^n$阶有限域存在且同构唯一\\
    $\ff{p^n} \cong \begin{cases}
            \ff{p}(\beta) = \ff{p}[x]/(f(x))          \\
            \{\sum a_i \beta^i | a_i \in \ff{p}\}     \\
            \{x^{p^n} = x \text{在$\ff{p}^{AC}$的全部根}\} \\
            \{0\} \cup \genfield{\alpha}, \alpha\text{为$\ff{p^n}$的本原元}
        \end{cases}$

    \textbf{本原元} $\ff{p^n}^*$中$o_\times(\alpha) = p^n - 1$

    \textbf{本原多项式\ } 本原元在$\ff{p}$上最小多项式

    \textbf{子域} $\ff{p^m} \leq \ff{p^n} \iff m | n$

    \begin{theorem}[多项式根]
        $f(x) \in \ff{q}[x]$为$n$次不可约,$u$为$\ff{q}^{AC}$一根,则全部根为$u, u^q,\cdots,u^{q^{n-1}}$
    \end{theorem}

    \begin{theorem}[一篮子分解]
        $x^{p^n} \! - \! x \! = \! \prod\limits_{d|n} g(x), g(x)$不可约
    \end{theorem}

    \section{典例}

    \begin{enumerate}
        \item 唯二四阶群$C_4$和Klein四元群
        \item $S_3$ 最小非交换群\\
              $\{(1), (123), (132)\} \nmsubgroupeq S_3$ \\
              $C((123)) = \{ (1), (123), (132) \}$ \\
              $N(\gengroup{(12)}) = \{ (1), (12) \}$
        \item $D_n = \{\rho^i \pi^j | 0 \leq i \leq n - 1, j = 0, 1 \}$ \\
              子群$\gengroup{\rho}, \gengroup{\pi}, \gengroup{\rho^k, \rho^l\pi} (k | n, 0 \!\leq\! l \!<\! k)$
        \item $S_n$中$1^{\lambda_1} 2^{\lambda_2}\cdots n^{\lambda_n}$型置换数\\
              $\frac{n!}{1^{\lambda_1} 2^{\lambda_2}\cdots n^{\lambda_n} \lambda_1! \lambda_2! \cdots \lambda_n!}$
        \item $K_4 \nmsubgroupeq S_4, S_4/K_4 \cong S_3$
        \item $n \geq 5, A_n$是单群
        \item UFD: $\setZ[x]$\\
              $\setZ[x]$中本原多项式如$x^3 + x^2 + 1$
        \item PID: $F[x]$
        \item ED: $\setZ, F[x], \setZ[i]$
        \item $\setC = \setR(i), (\setC : \setR) = 2$
        \item $\ff{2^2} \! = \! \{ 0, 1, \alpha, \alpha^2 \}$有3次本原多项式$x^3 + x + 1$和$x^3 + x^2 + 1$
    \end{enumerate}

\end{multicols}

\end{document}