\documentclass[a4paper,10pt]{ctexart}
\usepackage[left=.5cm, right=.5cm, top=.5cm, bottom=.5cm]{geometry}
\usepackage{multicol}

\usepackage{amsmath,amssymb,amsthm}
\usepackage{mathrsfs}
\usepackage{thmtools}

\usepackage{booktabs}
\usepackage[shortlabels]{enumitem}

\usepackage[colorlinks]{hyperref}
\usepackage{fontawesome5}

\usepackage{lipsum}

% ctex 排版设置
\ctexset{
    section = {
        beforeskip = {0ex},
        afterskip = {0ex},
    },
    subsection = {
        beforeskip = {0ex},
        afterskip = {0ex},
    },
    subsubsection = {
        beforeskip = {0ex},
        afterskip = {0ex},
    },
}
\setlength{\parindent}{0pt} % 抑制首行缩进以减少空白
\setlist{nosep} % 抑制列表项之间的空白

% 数学符号
\newcommand*{\bb}[1]{\mathbb{#1}}
\newcommand*{\setN}{\mathbb{N}}
\newcommand*{\setZ}{\mathbb{Z}}
\newcommand*{\setQ}{\mathbb{Q}}
\newcommand*{\setR}{\mathbb{R}}
\newcommand*{\setC}{\mathbb{C}}
\newcommand*{\st}{\text{ s.t. }}
\newcommand*{\powerset}[1]{\mathscr{P}(#1)}
\newcommand*{\impl}{\Rightarrow}
\renewcommand*{\iff}{\Leftrightarrow}
\renewcommand*{\leq}{\leqslant}
\renewcommand*{\geq}{\geqslant}
\newcommand*{\nmsubgroupeq}{\trianglelefteq}
\newcommand*{\nmsubgroup}{\triangleleft}
\newcommand*{\gengroup}[1]{\langle #1 \rangle}
\newcommand*{\genring}[1]{[#1]}
\newcommand*{\genideal}[1]{(#1)}
\newcommand*{\genfield}[1]{\langle #1 \rangle}
\newcommand{\ch}[1]{\text{ch}(#1)}
\newcommand{\set}[1]{\{#1\}}

% 数学环境
\declaretheoremstyle[
    spaceabove=0pt, spacebelow=0pt,
    headindent=-4pt,
    notefont=\bfseries,notebraces={}{}
]{tight-thm}
\declaretheorem[name=, numbered=no, style=tight-thm]{theorem}

\begin{document}

\begin{multicols}{3}
    \begin{center}
        {\Large 应用近世代数}

        暮月

        \href{https://github.com/duskmoon314/thu-ama-cheatsheet}{\faGithub \  duskmoon314/thu-ama-cheatsheet}
    \end{center}

    \section{预备知识}

    \subsection{集合与映射}

    \textbf{常见数集} $\setN, \setZ, \setQ, \setR, \setC$

    \textbf{幂集} A所有子集构成集合,$\powerset{A}$或$2^A$

    \textbf{映射} $f: A \to B$,$f$是A到B的映射

    \begin{itemize}
        \item[\emph{单射}] $\forall x_1, x_2 \!\in\! A, x_1 \!\neq\! x_2 \!\impl\! f(x_1) \!\neq\! f(x_2) $
        \item[\emph{满射}] $\forall y \in B, \exists x \in A, f(x) = y$
        \item[\emph{双射}] 单射且满射
    \end{itemize}

    \textbf{变换} $f: A \to A$,$f$是A上的变换

    \textbf{复合映射} $fg(x) = f(g(x))$

    \textbf{恒等变换} $I_A(x) = x$

    \textbf{映射的逆} $gf=I_A, fh=I_B$;若同时存在,则$g=h=f^{-1}$

    \subsection{二元关系}

    \textbf{二元运算} $f: S \times S \to S$

    \textbf{二元关系} $\forall a \!\in\! A, b \!\in\! B$可判$\!aRb\!$或$\!aR'b\!$

    \textbf{等价关系} $R$是等价关系$\iff$满足

    \begin{itemize}
        \item[\emph{自反}] $\forall a \!\in\! A, aRa$
        \item[\emph{对称}] $\forall a, b \!\in\! A, aRb \impl bRa$
        \item[\emph{传递}] $\forall a, b, c \!\in\! A, aRb, bRc \impl aRc $
    \end{itemize}

    \textbf{等价类} $\bar{a} = \{ x | x \in A, xRa \}$ \textbf{代表元}$a$

    \textbf{划分} $\{ A_\alpha | \alpha \in I \}$ 是划分$\iff$满足

    \begin{itemize}
        \item $\bigcup\limits_{\alpha \in I} A_\alpha = A$
        \item $\forall \alpha, \beta \in I, \alpha \ne \beta, A_\alpha \bigcap A_\beta = \emptyset$
    \end{itemize}

    \begin{theorem}[等价与划分]
        $aRb \iff \exists \alpha \in I, a, b \in A_\alpha$
    \end{theorem}

    \textbf{偏序} 自反、反对称、传递

    \textbf{全序}

    \subsection{整数与同余}

    \section{群论}

    \subsection{概念与例子}

    \textbf{群} $(G, \cdot)$ 封结幺逆交

    \begin{theorem}[幺元性质]
        $e_L = e_R = e$
    \end{theorem}

    \begin{theorem}[逆元性质]
        $\forall a \in G, a_L^{-1} = a_R^{-1} = a^{-1}$

        \begin{itemize}
            \item $(a^{-1})^{-1} = a$
            \item $(ab)^{-1}=b^{-1}a^{-1}$
            \item $(a^n)^{-1}=(a^{-1})^n=a^{-n}$
        \end{itemize}
    \end{theorem}

    \subsection{内部结构}

    \textbf{子群} $\emptyset \!\neq\! H \!\subseteq\! G$,封结幺逆$\impl H \!\leq\! G$

    \textbf{平凡子群} $\{e\}, G$

    \begin{theorem}[子群判则]
        以下等价
        \begin{enumerate}
            \item $H \leq G$
            \item $\forall a, b \!\in\! H$,$ab \!\in\! H, a^{-1} \!\in\! H$ (四合二)
            \item $\forall a, b \!\in\! H$,$ab^{-1} \!\in\! H$ (四合一)
        \end{enumerate}
    \end{theorem}

    \begin{theorem}[子群运算律]
        \hfil

        \begin{enumerate}
            \item $H \!\leq\! G$,则$H$中幺元为$G$中幺元
            \item $H_1, H_2 \!\leq\! G$,则$H_1 \!\cap\! H_2 \!\leq\! G$
            \item $H_1, H_2 \!\leq\! G$,则$H_1 \!\cup\! H_2 \!\leq\! G \!\iff\! H_1 \!\subseteq\! H_2 $或$H_2 \!\subseteq\! H_1$
            \item $H_1, H_2 \!\leq\! G$,则$H_1 H_2 \!\leq\! G \!\iff\! H_1 H_2 = H_2 H_1 \!\triangleq\! \{h_1h_2 | h_1 \!\in\! H_1, h_2 \!\in\! H_2\}$
        \end{enumerate}
    \end{theorem}

    \textbf{元素的阶} $a \!\in\! G$,最小$n$使$a^n \!=\! e$,记$o(a)$

    \begin{theorem}[阶的性质]
        \hfil

        \begin{enumerate}
            \item $o(e) = 1$, $o(a^{-1}) = o(a)$
            \item $a^m = e \iff o(a) | m$
            \item 有限群中所有元素的阶均有限,而无限群不一定存在无限阶元
            \item 设$o(a) = m, o(b) = n$,若$(m, n) = 1$且$ab = ba$,则$o(ab) = mn$
            \item $\forall a \!\in\! G \!\setminus\! \{e\}$,若$o(a) = 2$,G交换群
        \end{enumerate}
    \end{theorem}

    \textbf{生成子群} $S \!\subseteq\! G$,$G$中含$S$的子群交,$\gengroup{S}$

    \textbf{循环子群} 由1个元素生成的子群

    \textbf{极小生成集} $G \! = \! \gengroup{S}$,且任何S的真子集的生成子群均不是G

    极小生成集的元素个数唯一

    \begin{theorem}[循环群性质]
        \hfil

        \begin{enumerate}
            \item 循环群必交换
            \item 循环子群的阶等于生成元的阶
            \item $\gengroup{a} \cong \begin{cases}
                          (\setZ, +)        & o(a)=\infty       \\
                          (\setZ/n\setZ, +) & o(a) = n < \infty \\
                      \end{cases}$
            \item 阶数相同的循环群必同构
            \item $(\setZ, +)\!$子群必为$m\setZ(m \!\in\! \setZ)$形式

                  $(\setZ/n\setZ,\! +)\!$子群同构于$\!(\setZ/d\setZ,\! +)\  d|n$
        \end{enumerate}
    \end{theorem}

    \textbf{变换群} 非空集合$\Omega$上所有可逆变换全体

    \textbf{置换群(对称群)} 有限集上的变换群$S_n$

    \textbf{置换的共轭} $\tau \sigma \tau^{-1} = \sigma_1$

    \textbf{轮换} $\sigma(i_k) = i_{((k+1))}$,其余不动点

    \textbf{对换(循环)} 长度为2的轮换

    \begin{theorem}[标准轮换分解]
        $\forall \sigma \in S_n$,可分解为唯一不相交轮换之积$r_1 r_2 \cdots r_s$
    \end{theorem}

    \textbf{奇(偶)置换} $\sigma$可表为奇(偶)个对换之积

    \textbf{交错群} $A_n = \{ S_n \text{中全体偶置换} \}$

    \begin{theorem}[Cayley定理]
        任一群$G$必同构于左乘变换群。有限群则$G \!\cong\! G' \!\leq\! S_n$
    \end{theorem}

    \textbf{左(右)陪集} $\forall a \!\in\! G, H \!\leq\! G, aH(Ha)$

    \begin{theorem}[陪集相等判则]
        $aH = bH \iff a^{-1}b \in H$
    \end{theorem}

    \textbf{指数} 左(右)陪集的个数,记$[G\!:\!H](H \!\leq\! G)$

    \begin{theorem}[Lagrange定理]
        $|G| \! < \! \infty, H \!\leq\! G \!\impl\! |G| \! =\! |H|[G:H]$
    \end{theorem}

    \begin{theorem}[子群阶]
        $A, B \!\leq\! G$且有限$\impl |AB| = \frac{ |A| |B| }{ |A \cap B| }$
    \end{theorem}

    \textbf{正规子群} $H \!\!\leq\!\! G, \forall g \!\in\! G, gH \! = \! Hg$记$H \!\nmsubgroupeq\! G$

    \textbf{单群} 不存在非平凡正规子群的群

    \begin{theorem}[正规子群判则]
        $H \leq G$,以下等价
        \begin{enumerate}
            \item $H \nmsubgroupeq G$
            \item $\forall g \in G,\forall h \in H, ghg^{-1} \in H$
            \item $\forall g \in G, gHg^{-1} \subseteq H$
            \item $\forall g \in G, gHg^{-1} = H$
        \end{enumerate}
    \end{theorem}

    \begin{theorem}[正规子群运算律]
        \hfil

        \begin{enumerate}
            \item $A \!\nmsubgroupeq\! G, B \!\nmsubgroupeq\! G \!\impl\! A \!\cap\! B \!\nmsubgroupeq\! G, AB \!\nmsubgroupeq\! G$
            \item $A \!\nmsubgroupeq\! G, B \!\leq\! G \!\impl\! A \!\cap\! B \!\nmsubgroupeq\! B, AB \!\leq\! G$
            \item $A, B \!\nmsubgroupeq\! G, A \!\cap\! B \! = \! \{e\} \!\impl\! ab \! = \! ba \\ (\forall a \!\in\! A, b \!\in\! B)$
        \end{enumerate}
    \end{theorem}

    \textbf{换位子} $aba^{-1}b^{-1}$,记作$[a,b]$

    \textbf{换位子群} $\{ \gengroup{aba^{-1}b^{-1}} | a, b \in G \} \nmsubgroupeq G$

    \textbf{商群} $H \!\nmsubgroupeq\! G$,$G/H$在子集乘法下构成群

    \begin{theorem}[素阶元]
        有限交换群$G$,素数$p|\ |G|$,$G$必有$p$阶元
    \end{theorem}

    \textbf{中心} $C(G) = \{ x \in G | xg = gx, \forall g \in G\}$

    \textbf{中心化子} $C_G(A) \! = \! \{x \!\in\! G | xa \! = \! ax, \forall a \!\in\! A\}$

    \begin{theorem}[中心化子性质]
        \hfil

        \begin{enumerate}
            \item $C_G(a) \cap C_G(b) = C_G(\{a, b\})$
            \item $\gengroup{a} \subseteq C_G(a) \impl \gengroup{a} \leq C_G(A)$
        \end{enumerate}
    \end{theorem}

    \textbf{共轭} $a, b \in G, \exists g \in G \st gag^{-1}=b$

    \begin{theorem}[共轭类]
        $|K_A| < \infty, |K_a| = [G:C_G(a)]$
    \end{theorem}

    \begin{theorem}[类方程]
        $|G| \! < \! \infty, |G| \! = \! |C| + \sum\limits_{a \notin C} [G:C_G(a)]$
    \end{theorem}

    \textbf{共轭子群} $H \leq G, H_1 = gHg^{-1}$

    \textbf{正规化子} $N_G(H) = \{g \in G | gH = Hg\}$

    \begin{theorem}[共轭子群类]
        $|K_H| \!\! < \!\! \infty, |K_H| \!\! = \!\! [G:N_G(H)]$
    \end{theorem}

    \subsection{外部联系}

    \textbf{同态} \ 映射 + 保运算

    \textbf{单同态} \ 单射 + 同态 $G \overset{f}{\hookrightarrow} G'$

    \textbf{满同态} \ 满射 + 同态 $G \overset{f}{\twoheadrightarrow} G'$

    \textbf{同构} \ 双射 + 同态 $G \overunderset{\sim}{f}{\rightarrow} G'$

    \begin{theorem}[同态性质]
        \hfil

        \begin{enumerate}
            \item 保幺元、逆元、方幂
            \item 保子群、反向局部保子群

            \item TODO
        \end{enumerate}
    \end{theorem}

    \begin{theorem}[同态基本定理]
        $f: G \rightarrow G', K = \ker f$

        \begin{enumerate}
            \item $G/K \cong f(G)$,若$f$满,则$G/K \cong G'$
            \item $\pi: G \twoheadrightarrow G/K, j: f(G) \hookrightarrow G' \impl f = j \circ \bar{f} \circ \pi, \bar{f}: G/K \rightarrow f(G), \bar{a} \mapsto f(a)$
        \end{enumerate}
    \end{theorem}

    \begin{theorem}[子群对应定理]
        TODO
    \end{theorem}

    \begin{theorem}[商群同构定理(第一同构定理)]
        TODO
    \end{theorem}

    \begin{theorem}[子群乘积同构定理(第二同构定理)]
        \hfil

        $N \nmsubgroupeq G, H \leq G \impl HN/N \cong H/(H \cap N)$
    \end{theorem}

    \subsection{群在集合上的作用}

    \textbf{群作用}

    \textbf{轨道}

    \textbf{不动点}

    \textbf{稳定化子}

    \begin{theorem}[Burnside 引理]

    \end{theorem}

    \section{环论}

    \subsection{概念与例子}

    \textbf{环} $(R, +, \cdot)$ 封结幺逆交 + 封结 + 分配律

    \textbf{零元} $0 \cdot a = a \cdot 0 (\forall a \in R)$

    \textbf{负元} $(-a)b = a(-b) = -(ab) (\forall a, b \in R)$

    \textbf{单位} $U(R)$ 可逆元

    \textbf{零因子} $a, b \in R, ab = 0 \wedge a, b \ne 0$

    \textbf{整环} $R\ne \{0\}$,含幺可交换无零因子

    \textbf{除环} $\{0, 1\} \subseteq R, R \setminus \{0\}$构成乘法群

    \subsection{内部结构}

    \textbf{子环} $\emptyset \!\neq\! S \!\subseteq\! R \wedge (S, +, \cdot)$为环

    \begin{theorem}[子环判则]

    \end{theorem}

    \begin{theorem}[子环运算律]

    \end{theorem}

    \textbf{左(右)理想} $\forall a \in I, r \in R, ra \in I (ar \in I)$

    \begin{theorem}[理想判则]

    \end{theorem}

    \textbf{单环} 只有平凡理想 $\{0\}$ 和 $R$ 的环

    \textbf{生成子环} $\emptyset \!\neq\! S \!\subseteq\! R, R$中含$\!S\!$的子环交$\genring{S}$

    \textbf{生成理想} $\emptyset \!\neq\! S \!\subseteq\! R, R$中含$\!S\!$的理想交$\genideal{S}$

    \textbf{主理想} 由1个元素生成的理想

    \textbf{商环}

    \textbf{极大理想}

    \subsection{外部联系}

    \subsection{特殊环}

    \section{域论}

    \subsection{域扩张}

    \textbf{子域/扩域} $\emptyset \ne F \subseteq K$,皆为域

    \textbf{生成子域} $S \subseteq F$,$F$中含$S$最小子域,$\genfield{S}$

    \textbf{素域} $F_0 = \genfield{1}$

    \textbf{特征} $\ch{F} = \begin{cases}
            p & \text{若}o^+(1) = p < \infty \\
            0 & \text{若}o^+(1) = \infty
        \end{cases}$

    \begin{theorem}[素域同构]
        $F_0$是$F$的素域,

        $F_0 \cong \begin{cases}
                \setQ        & o^+(1) = \infty     \\
                \setZ/p\setZ & o^+(1) = p < \infty
            \end{cases}$
    \end{theorem}

    \textbf{扩张次数} $K/F$,$K$作为$F$-线性空间的维数$\dim_F K$,$(K:F)$

    \begin{theorem}[链式法则]
        $F\subseteq K \subseteq E, E/F$有限扩张$\iff (E:F)=(E:K)(K:F)$
    \end{theorem}

    \textbf{添元扩张} $K/F, S\subseteq K$,含$S\cup F$的$K$最小子域为$F(S)$

    \textbf{单扩张} $S = \set{u}, F(u)$

    \textbf{代数元(超越元)} $K/F, u\in K, u$是(否) $F$上某非零多项式$f(x)\in F[x]$根

    \begin{theorem}[单扩张定理]

    \end{theorem}

    \textbf{代数扩张(超越扩张)}

    \textbf{代数闭包}

\end{multicols}

\end{document}