\documentclass[a4paper,10pt]{ctexart}
\usepackage[left=.5cm, right=.5cm, top=.5cm, bottom=.5cm]{geometry}
\usepackage{multicol}

\usepackage{amsmath,amssymb,amsthm}
\usepackage{mathrsfs}

\usepackage{booktabs}
\usepackage[shortlabels]{enumitem}

\usepackage[colorlinks]{hyperref}
\usepackage{fontawesome5}

\usepackage{lipsum}

% ctex 排版设置
\ctexset{
    section = {
        beforeskip = {0ex},
        afterskip = {0ex},
    },
    subsection = {
        beforeskip = {0ex},
        afterskip = {0ex},
    },
    subsubsection = {
        beforeskip = {0ex},
        afterskip = {0ex},
    },
}
\setlength{\parindent}{0pt} % 抑制首行缩进以减少空白
\setlist{nosep} % 抑制列表项之间的空白

% 数学符号
\newcommand*{\setN}{\mathbb{N}}
\newcommand*{\setZ}{\mathbb{Z}}
\newcommand*{\setQ}{\mathbb{Q}}
\newcommand*{\setR}{\mathbb{R}}
\newcommand*{\setC}{\mathbb{C}}
\newcommand*{\powerset}[1]{\mathscr{P}(#1)}
\newcommand*{\impl}{\Rightarrow}

\begin{document}

\begin{multicols}{3}
    \begin{center}
        {\Large 应用近世代数}

        暮月

        \href{https://github.com/duskmoon314/thu-ama-cheatsheet}{\faGithub \  duskmoon314/thu-ama-cheatsheet}
    \end{center}

    \section{预备知识}

    \subsection{集合与映射}

    \textbf{常见数集} $\setN, \setZ, \setQ, \setR, \setC$

    \textbf{幂集} A所有子集构成集合,$\powerset{A}$或$2^A$

    \textbf{映射} $f: A \to B$,$f$是A到B的映射

    \begin{itemize}
        \item[\emph{单射}] $\forall x_1, x_2 \!\in\! A, x_1 \!\neq\! x_2 \!\impl\! f(x_1) \!\neq\! f(x_2) $
        \item[\emph{满射}] $\forall y \in B, \exists x \in A, f(x) = y$
        \item[\emph{双射}] 单射且满射
    \end{itemize}

    \textbf{变换} $f: A \to A$,$f$是A上的变换

    \textbf{复合映射} $fg(x) = f(g(x))$

    \textbf{恒等变换} $I_A(x) = x$

    \textbf{映射的逆} $gf=I_A, fh=I_B$;若同时存在,则$g=h=f^{-1}$

    \subsection{二元关系}

    \subsection{整数与同余}

    \section{群论}

    \section{环论}

    \section{域论}

\end{multicols}

\end{document}